% coding:utf-8

\section{Maple}

\subsection{Dezimalbereich}
Um in Maple den Dezimalbereich zu definieren benutzt man eine Zuweisung
an eine interne Variabel namens \verb!Digits!. Dieser kann ein Integer-Wert
zugewiesen werden. Der Default-Wert ist 10. Für beispielsweise ein 
Dezimalstellenbereich von 5 zu haben gibt man folgendes ein.

\begin{lstlisting}[caption=Dezimalstellen, label=Dezimalstellen]
Digits:=5
\end{lstlisting}

\subsection{Numerische Werte}
Um in Maple einen Wert numerisch anzuzeigen kann die Funktion \verb!evalf()!
verwendet werden.

\begin{lstlisting}[caption=Numerisch, label=Numerisch]
evalf()
\end{lstlisting}

\subsection{Intervalle}
Intervalle werden im Maple in der folgenden Form angegeben: Bsp. $[a,b]$

\begin{lstlisting}[caption=Intervalle, label=Intervalle]
a..b
\end{lstlisting}

\subsection{Buffer verwenden}
Rechnet man in Maple etwas aus, so ist die jeweils letzte Ausgabe von
Maple in einem Buffer abgelegt. Das im Buffer abgelegte Ergebnis kann 
verwendet werden mit der dafür reservierten Variable \verb!%!. Möchte man
das zuvor gerechnete Ergebnis mal 2 rechnen so gibt man folgendes ein.

\begin{lstlisting}[caption=Buffer lesen, label=Buffer lesen]
2*%
\end{lstlisting}

\subsection{Vereinfachen}
Um einen Term im Maple zu vereinfachen kann die Funktion \verb!simplify()!
genutzt werden.

\begin{lstlisting}[caption=Vereinfachen, label=Vereinfachen]
simplify()
\end{lstlisting}

\subsection{Faktorisieren}
Um einen Term zu Faktorisieren kann \verb!factor()! genutzt werden.

\begin{lstlisting}[caption=Faktorisieren, label=Faktorisieren]
factor()
\end{lstlisting}

\subsection{Ausmultiplizieren}
Um einen Term in ausmultiplizierter Form zu erhalten kann \verb!expand()!
genutzt werden.

\begin{lstlisting}[caption=Ausmultiplizieren, label=Ausmultiplizieren]
expand()
\end{lstlisting}

\subsection{Kreiszahl}
Um in Maple die Kreiszahl $\pi$ zu benutzen muss speziell auf die Syntax
geachtet werden da hier leicht Fehler entstehen die nicht sofort bemerkt 
werden. Will man mit der Kreiszahl $\pi$ rechnen so muss \verb!Pi! 
mit grossem \verb!P! eingegeben werden. Schreibt man es klein \verb!pi!
so meint Maple, man wollte den griechischen Buchstaben und nicht die 
Kreiszahl benutzen.

\begin{lstlisting}[caption=Pi, label=Pi]
Pi
\end{lstlisting}

\subsection{Euler'sche Zahl}
In Maple muss die Eulersche Zahl als \verb!exp()! und nicht etwa als 
\verb!e! geschrieben werden. Der Term $2 \cdot x + 5 \cdot e^{2 \cdot x}$
wird wie folgt eingegeben.

\begin{lstlisting}[caption=Euler, label=Euler]
2*x+5*exp(2*x)
\end{lstlisting}


\subsection{Annahmen}
Mit \verb!assume()! können Annahmen zugewiesen werden wie $n = 10$, 
$n \in \mathbb{Z}$ usw. Hier einige Beispiele in Maple-Syntax.

\begin{lstlisting}[caption=Annahmen, label=Annahmen]
assume(n,'integer')
assume(n,'natural')
assume(n=10)
assume(n>10)
\end{lstlisting}

\noindent
\textbf{Achtung}: Es sollte wo immer möglich eine fixe Definition statt
eines \verb!assume()! verwendet werden da diese nicht zwingend zu Gleichen 
Ergebnissen führen. D.h. wenn man $n=10$ haben möchte, sollte man 
\verb!n:=10! und nicht \verb!assume(n=10)! verwenden.

\subsection{Funktionen}
In Maple müssen Funktionen immer wie folgt eingegeben werden
\begin{enumerate}
	\item Funktionsname (f)
	\item Zuweisung (:=)
	\item unabhängige Variable (x)
	\item Funktionszuweisung (->)
\end{enumerate}

$f(x)=2 \cdot x + 3$

\begin{lstlisting}[caption=Funktionen, label=Funktionen]
f:=x->2*x+3
\end{lstlisting}

\subsubsection{Stückweise Definitionen}
Man kann im Maple Funktionen auch Stükweise definieren mit \verb!pieceweise!.
Die Symtax hat dabei die folgende Form.

\begin{lstlisting}[caption=Stückweise Definition, label=Stückweise Definition]
f:=x->pieceweise(x<0, 2*x, x=0, 0, x>0, -2*x)
\end{lstlisting}

\noindent
Das obige Beispiel entspricht dabei der folgenden Funktion:

\begin{displaymath}
	f(x) = \left\{
		\begin{array}{lr}
			2 \cdot x &  x < 0 \\
			0 & x = 0 \\
			-2 \cdot x & x > 0
		\end{array}
        \right.
\end{displaymath}


\subsubsection{Summen}
Die Funktion $s(t)=\sum\limits_{k=1}^n k \cdot cos(k \cdot t) $ wird wie folgt
eingegeben.
\begin{lstlisting}[caption=Summen, label=Summen]
s:=t->sum(k*cos(k*t),k=1..n)
\end{lstlisting}

\subsection{Gleichungen lösen}

\subsubsection{Symbolisch}
Möchte man eine Gleichung symbolisch lösen so verwendet man die Funktion
\verb!solve()!.

\begin{lstlisting}[caption=Gleichung symbolisch, label=Gleichungen symbolisch I]
solve(2*x+5=0,x)
\end{lstlisting}

\noindent
Mit dieser lassen sich auch Gleichungssyteme lösen. Hierzu werden die 
Gleichungen in eckige Klammern gesetzt und per Komma getrennt. Genau gleich
die gesuchten Parameter. Eine typische Aufgabe ist die folgende.

\begin{lstlisting}[caption=Gleichungssystem, label=Gleichungssystem]
f1:=x->a*x^3+b*x^2+c*x+d
f2:=D^(1)(f1)
f3:=D^(2)(f1)
f4:=D^(3)(f1)
solve([f2(0)=0, f3(2)=0, f4(-2)=0], [a,b,c,d])
\end{lstlisting}

\subsubsection{Numerisch}
Analog zum symbolischen lösen von Gleichungen und Gleichungssystemen wird
für die numerischen die Funktion \verb!fsolve()! verwendet.

\begin{lstlisting}[caption=Gleichung numerisch, label=Gleichung numerisch]
fsolve()
\end{lstlisting}

\subsection{Parameter zuweisen}
Hat man zum Beispiel ein Gleichungssystem gelöst, erhält man symbolische oder
numerische Werte für diese. Möchte man diese gleich Zuweisen, kann die Funktion
\verb!assign()! verwendet werden.

\begin{lstlisting}[caption=Parameter zuweisen, label=Parameter zuweisen]
solve( ... , [a,b,c,d])
assign(%)
\end{lstlisting}

\subsection{Plotten}
In Maple gibt es 3 Typen von Plots:
\begin{itemize}
	\item \verb!plot()! \hfill gewöhnlicher Plot
	\item \verb!implicitplot()! \hfill impliziter Plot
	\item \verb!plot3d()! \hfill dreidimensionaler Plot
\end{itemize}

\subsubsection{Gewöhnlicher Plot}
Beim gewöhnlichen Plot kann eine Menge von Parametern definiert werden.
Im folgenden ist ein Minimal und Maximalbeispiel zu sehen.

\begin{lstlisting}[caption=Plot, label=Plot]
f1:=x->10+5*sin((2*Pi)/3+((5*Pi)/2))
f2:=x-> ... 
f3:=x-> ...

plot(f1, x)

plot([f1(x),f2(x),f3(x)], 
     x=-10..10,
     color=[red, blue, green],
     thickness=[3, 5, 7],
     legend=["Legend 1", "Legend 2", "Legend 3"],
     title="Sinusfunktion",
     xtickmarks=10,
     ytickmarks=[2.5='A', 4.5='B'],
     grid=[a,b],
     axes=boxed,
     ...
     )


\end{lstlisting}
